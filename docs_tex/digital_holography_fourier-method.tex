\documentclass[11pt]{article}

    \usepackage[breakable]{tcolorbox}
    \usepackage{parskip} % Stop auto-indenting (to mimic markdown behaviour)
    

    % Basic figure setup, for now with no caption control since it's done
    % automatically by Pandoc (which extracts ![](path) syntax from Markdown).
    \usepackage{graphicx}
    % Maintain compatibility with old templates. Remove in nbconvert 6.0
    \let\Oldincludegraphics\includegraphics
    % Ensure that by default, figures have no caption (until we provide a
    % proper Figure object with a Caption API and a way to capture that
    % in the conversion process - todo).
    \usepackage{caption}
    \DeclareCaptionFormat{nocaption}{}
    \captionsetup{format=nocaption,aboveskip=0pt,belowskip=0pt}

    \usepackage{float}
    \floatplacement{figure}{H} % forces figures to be placed at the correct location
    \usepackage{xcolor} % Allow colors to be defined
    \usepackage{enumerate} % Needed for markdown enumerations to work
    \usepackage{geometry} % Used to adjust the document margins
    \usepackage{amsmath} % Equations
    \usepackage{amssymb} % Equations
    \usepackage{textcomp} % defines textquotesingle
    % Hack from http://tex.stackexchange.com/a/47451/13684:
    \AtBeginDocument{%
        \def\PYZsq{\textquotesingle}% Upright quotes in Pygmentized code
    }
    \usepackage{upquote} % Upright quotes for verbatim code
    \usepackage{eurosym} % defines \euro

    \usepackage{iftex}
    \ifPDFTeX
        \usepackage[T1]{fontenc}
        \IfFileExists{alphabeta.sty}{
              \usepackage{alphabeta}
          }{
              \usepackage[mathletters]{ucs}
              \usepackage[utf8x]{inputenc}
          }
    \else
        \usepackage{fontspec}
        \usepackage{unicode-math}
    \fi

    \usepackage{fancyvrb} % verbatim replacement that allows latex
    \usepackage{grffile} % extends the file name processing of package graphics
                         % to support a larger range
    \makeatletter % fix for old versions of grffile with XeLaTeX
    \@ifpackagelater{grffile}{2019/11/01}
    {
      % Do nothing on new versions
    }
    {
      \def\Gread@@xetex#1{%
        \IfFileExists{"\Gin@base".bb}%
        {\Gread@eps{\Gin@base.bb}}%
        {\Gread@@xetex@aux#1}%
      }
    }
    \makeatother
    \usepackage[Export]{adjustbox} % Used to constrain images to a maximum size
    \adjustboxset{max size={ 0.7\linewidth}{ 0.7\paperheight}}

    % The hyperref package gives us a pdf with properly built
    % internal navigation ('pdf bookmarks' for the table of contents,
    % internal cross-reference links, web links for URLs, etc.)
    \usepackage{hyperref}
    % The default LaTeX title has an obnoxious amount of whitespace. By default,
    % titling removes some of it. It also provides customization options.
    \usepackage{titling}
    \usepackage{longtable} % longtable support required by pandoc >1.10
    \usepackage{booktabs}  % table support for pandoc > 1.12.2
    \usepackage{array}     % table support for pandoc >= 2.11.3
    \usepackage{calc}      % table minipage width calculation for pandoc >= 2.11.1
    \usepackage[inline]{enumitem} % IRkernel/repr support (it uses the enumerate* environment)
    \usepackage[normalem]{ulem} % ulem is needed to support strikethroughs (\sout)
                                % normalem makes italics be italics, not underlines
    \usepackage{mathrsfs}
    

    
    % Colors for the hyperref package
    \definecolor{urlcolor}{rgb}{0,.145,.698}
    \definecolor{linkcolor}{rgb}{.71,0.21,0.01}
    \definecolor{citecolor}{rgb}{.12,.54,.11}

    % ANSI colors
    \definecolor{ansi-black}{HTML}{3E424D}
    \definecolor{ansi-black-intense}{HTML}{282C36}
    \definecolor{ansi-red}{HTML}{E75C58}
    \definecolor{ansi-red-intense}{HTML}{B22B31}
    \definecolor{ansi-green}{HTML}{00A250}
    \definecolor{ansi-green-intense}{HTML}{007427}
    \definecolor{ansi-yellow}{HTML}{DDB62B}
    \definecolor{ansi-yellow-intense}{HTML}{B27D12}
    \definecolor{ansi-blue}{HTML}{208FFB}
    \definecolor{ansi-blue-intense}{HTML}{0065CA}
    \definecolor{ansi-magenta}{HTML}{D160C4}
    \definecolor{ansi-magenta-intense}{HTML}{A03196}
    \definecolor{ansi-cyan}{HTML}{60C6C8}
    \definecolor{ansi-cyan-intense}{HTML}{258F8F}
    \definecolor{ansi-white}{HTML}{C5C1B4}
    \definecolor{ansi-white-intense}{HTML}{A1A6B2}
    \definecolor{ansi-default-inverse-fg}{HTML}{FFFFFF}
    \definecolor{ansi-default-inverse-bg}{HTML}{000000}

    % common color for the border for error outputs.
    \definecolor{outerrorbackground}{HTML}{FFDFDF}

    % commands and environments needed by pandoc snippets
    % extracted from the output of `pandoc -s`
    \providecommand{\tightlist}{%
      \setlength{\itemsep}{0pt}\setlength{\parskip}{0pt}}
    \DefineVerbatimEnvironment{Highlighting}{Verbatim}{commandchars=\\\{\}}
    % Add ',fontsize=\small' for more characters per line
    \newenvironment{Shaded}{}{}
    \newcommand{\KeywordTok}[1]{\textcolor[rgb]{0.00,0.44,0.13}{\textbf{{#1}}}}
    \newcommand{\DataTypeTok}[1]{\textcolor[rgb]{0.56,0.13,0.00}{{#1}}}
    \newcommand{\DecValTok}[1]{\textcolor[rgb]{0.25,0.63,0.44}{{#1}}}
    \newcommand{\BaseNTok}[1]{\textcolor[rgb]{0.25,0.63,0.44}{{#1}}}
    \newcommand{\FloatTok}[1]{\textcolor[rgb]{0.25,0.63,0.44}{{#1}}}
    \newcommand{\CharTok}[1]{\textcolor[rgb]{0.25,0.44,0.63}{{#1}}}
    \newcommand{\StringTok}[1]{\textcolor[rgb]{0.25,0.44,0.63}{{#1}}}
    \newcommand{\CommentTok}[1]{\textcolor[rgb]{0.38,0.63,0.69}{\textit{{#1}}}}
    \newcommand{\OtherTok}[1]{\textcolor[rgb]{0.00,0.44,0.13}{{#1}}}
    \newcommand{\AlertTok}[1]{\textcolor[rgb]{1.00,0.00,0.00}{\textbf{{#1}}}}
    \newcommand{\FunctionTok}[1]{\textcolor[rgb]{0.02,0.16,0.49}{{#1}}}
    \newcommand{\RegionMarkerTok}[1]{{#1}}
    \newcommand{\ErrorTok}[1]{\textcolor[rgb]{1.00,0.00,0.00}{\textbf{{#1}}}}
    \newcommand{\NormalTok}[1]{{#1}}

    % Additional commands for more recent versions of Pandoc
    \newcommand{\ConstantTok}[1]{\textcolor[rgb]{0.53,0.00,0.00}{{#1}}}
    \newcommand{\SpecialCharTok}[1]{\textcolor[rgb]{0.25,0.44,0.63}{{#1}}}
    \newcommand{\VerbatimStringTok}[1]{\textcolor[rgb]{0.25,0.44,0.63}{{#1}}}
    \newcommand{\SpecialStringTok}[1]{\textcolor[rgb]{0.73,0.40,0.53}{{#1}}}
    \newcommand{\ImportTok}[1]{{#1}}
    \newcommand{\DocumentationTok}[1]{\textcolor[rgb]{0.73,0.13,0.13}{\textit{{#1}}}}
    \newcommand{\AnnotationTok}[1]{\textcolor[rgb]{0.38,0.63,0.69}{\textbf{\textit{{#1}}}}}
    \newcommand{\CommentVarTok}[1]{\textcolor[rgb]{0.38,0.63,0.69}{\textbf{\textit{{#1}}}}}
    \newcommand{\VariableTok}[1]{\textcolor[rgb]{0.10,0.09,0.49}{{#1}}}
    \newcommand{\ControlFlowTok}[1]{\textcolor[rgb]{0.00,0.44,0.13}{\textbf{{#1}}}}
    \newcommand{\OperatorTok}[1]{\textcolor[rgb]{0.40,0.40,0.40}{{#1}}}
    \newcommand{\BuiltInTok}[1]{{#1}}
    \newcommand{\ExtensionTok}[1]{{#1}}
    \newcommand{\PreprocessorTok}[1]{\textcolor[rgb]{0.74,0.48,0.00}{{#1}}}
    \newcommand{\AttributeTok}[1]{\textcolor[rgb]{0.49,0.56,0.16}{{#1}}}
    \newcommand{\InformationTok}[1]{\textcolor[rgb]{0.38,0.63,0.69}{\textbf{\textit{{#1}}}}}
    \newcommand{\WarningTok}[1]{\textcolor[rgb]{0.38,0.63,0.69}{\textbf{\textit{{#1}}}}}


    % Define a nice break command that doesn't care if a line doesn't already
    % exist.
    \def\br{\hspace*{\fill} \\* }
    % Math Jax compatibility definitions
    \def\gt{>}
    \def\lt{<}
    \let\Oldtex\TeX
    \let\Oldlatex\LaTeX
    \renewcommand{\TeX}{\textrm{\Oldtex}}
    \renewcommand{\LaTeX}{\textrm{\Oldlatex}}
    % Document parameters
    % Document title
    \title{Off-axis Digital Holography}
    
    
    
    
    
% Pygments definitions
\makeatletter
\def\PY@reset{\let\PY@it=\relax \let\PY@bf=\relax%
    \let\PY@ul=\relax \let\PY@tc=\relax%
    \let\PY@bc=\relax \let\PY@ff=\relax}
\def\PY@tok#1{\csname PY@tok@#1\endcsname}
\def\PY@toks#1+{\ifx\relax#1\empty\else%
    \PY@tok{#1}\expandafter\PY@toks\fi}
\def\PY@do#1{\PY@bc{\PY@tc{\PY@ul{%
    \PY@it{\PY@bf{\PY@ff{#1}}}}}}}
\def\PY#1#2{\PY@reset\PY@toks#1+\relax+\PY@do{#2}}

\@namedef{PY@tok@w}{\def\PY@tc##1{\textcolor[rgb]{0.73,0.73,0.73}{##1}}}
\@namedef{PY@tok@c}{\let\PY@it=\textit\def\PY@tc##1{\textcolor[rgb]{0.24,0.48,0.48}{##1}}}
\@namedef{PY@tok@cp}{\def\PY@tc##1{\textcolor[rgb]{0.61,0.40,0.00}{##1}}}
\@namedef{PY@tok@k}{\let\PY@bf=\textbf\def\PY@tc##1{\textcolor[rgb]{0.00,0.50,0.00}{##1}}}
\@namedef{PY@tok@kp}{\def\PY@tc##1{\textcolor[rgb]{0.00,0.50,0.00}{##1}}}
\@namedef{PY@tok@kt}{\def\PY@tc##1{\textcolor[rgb]{0.69,0.00,0.25}{##1}}}
\@namedef{PY@tok@o}{\def\PY@tc##1{\textcolor[rgb]{0.40,0.40,0.40}{##1}}}
\@namedef{PY@tok@ow}{\let\PY@bf=\textbf\def\PY@tc##1{\textcolor[rgb]{0.67,0.13,1.00}{##1}}}
\@namedef{PY@tok@nb}{\def\PY@tc##1{\textcolor[rgb]{0.00,0.50,0.00}{##1}}}
\@namedef{PY@tok@nf}{\def\PY@tc##1{\textcolor[rgb]{0.00,0.00,1.00}{##1}}}
\@namedef{PY@tok@nc}{\let\PY@bf=\textbf\def\PY@tc##1{\textcolor[rgb]{0.00,0.00,1.00}{##1}}}
\@namedef{PY@tok@nn}{\let\PY@bf=\textbf\def\PY@tc##1{\textcolor[rgb]{0.00,0.00,1.00}{##1}}}
\@namedef{PY@tok@ne}{\let\PY@bf=\textbf\def\PY@tc##1{\textcolor[rgb]{0.80,0.25,0.22}{##1}}}
\@namedef{PY@tok@nv}{\def\PY@tc##1{\textcolor[rgb]{0.10,0.09,0.49}{##1}}}
\@namedef{PY@tok@no}{\def\PY@tc##1{\textcolor[rgb]{0.53,0.00,0.00}{##1}}}
\@namedef{PY@tok@nl}{\def\PY@tc##1{\textcolor[rgb]{0.46,0.46,0.00}{##1}}}
\@namedef{PY@tok@ni}{\let\PY@bf=\textbf\def\PY@tc##1{\textcolor[rgb]{0.44,0.44,0.44}{##1}}}
\@namedef{PY@tok@na}{\def\PY@tc##1{\textcolor[rgb]{0.41,0.47,0.13}{##1}}}
\@namedef{PY@tok@nt}{\let\PY@bf=\textbf\def\PY@tc##1{\textcolor[rgb]{0.00,0.50,0.00}{##1}}}
\@namedef{PY@tok@nd}{\def\PY@tc##1{\textcolor[rgb]{0.67,0.13,1.00}{##1}}}
\@namedef{PY@tok@s}{\def\PY@tc##1{\textcolor[rgb]{0.73,0.13,0.13}{##1}}}
\@namedef{PY@tok@sd}{\let\PY@it=\textit\def\PY@tc##1{\textcolor[rgb]{0.73,0.13,0.13}{##1}}}
\@namedef{PY@tok@si}{\let\PY@bf=\textbf\def\PY@tc##1{\textcolor[rgb]{0.64,0.35,0.47}{##1}}}
\@namedef{PY@tok@se}{\let\PY@bf=\textbf\def\PY@tc##1{\textcolor[rgb]{0.67,0.36,0.12}{##1}}}
\@namedef{PY@tok@sr}{\def\PY@tc##1{\textcolor[rgb]{0.64,0.35,0.47}{##1}}}
\@namedef{PY@tok@ss}{\def\PY@tc##1{\textcolor[rgb]{0.10,0.09,0.49}{##1}}}
\@namedef{PY@tok@sx}{\def\PY@tc##1{\textcolor[rgb]{0.00,0.50,0.00}{##1}}}
\@namedef{PY@tok@m}{\def\PY@tc##1{\textcolor[rgb]{0.40,0.40,0.40}{##1}}}
\@namedef{PY@tok@gh}{\let\PY@bf=\textbf\def\PY@tc##1{\textcolor[rgb]{0.00,0.00,0.50}{##1}}}
\@namedef{PY@tok@gu}{\let\PY@bf=\textbf\def\PY@tc##1{\textcolor[rgb]{0.50,0.00,0.50}{##1}}}
\@namedef{PY@tok@gd}{\def\PY@tc##1{\textcolor[rgb]{0.63,0.00,0.00}{##1}}}
\@namedef{PY@tok@gi}{\def\PY@tc##1{\textcolor[rgb]{0.00,0.52,0.00}{##1}}}
\@namedef{PY@tok@gr}{\def\PY@tc##1{\textcolor[rgb]{0.89,0.00,0.00}{##1}}}
\@namedef{PY@tok@ge}{\let\PY@it=\textit}
\@namedef{PY@tok@gs}{\let\PY@bf=\textbf}
\@namedef{PY@tok@gp}{\let\PY@bf=\textbf\def\PY@tc##1{\textcolor[rgb]{0.00,0.00,0.50}{##1}}}
\@namedef{PY@tok@go}{\def\PY@tc##1{\textcolor[rgb]{0.44,0.44,0.44}{##1}}}
\@namedef{PY@tok@gt}{\def\PY@tc##1{\textcolor[rgb]{0.00,0.27,0.87}{##1}}}
\@namedef{PY@tok@err}{\def\PY@bc##1{{\setlength{\fboxsep}{\string -\fboxrule}\fcolorbox[rgb]{1.00,0.00,0.00}{1,1,1}{\strut ##1}}}}
\@namedef{PY@tok@kc}{\let\PY@bf=\textbf\def\PY@tc##1{\textcolor[rgb]{0.00,0.50,0.00}{##1}}}
\@namedef{PY@tok@kd}{\let\PY@bf=\textbf\def\PY@tc##1{\textcolor[rgb]{0.00,0.50,0.00}{##1}}}
\@namedef{PY@tok@kn}{\let\PY@bf=\textbf\def\PY@tc##1{\textcolor[rgb]{0.00,0.50,0.00}{##1}}}
\@namedef{PY@tok@kr}{\let\PY@bf=\textbf\def\PY@tc##1{\textcolor[rgb]{0.00,0.50,0.00}{##1}}}
\@namedef{PY@tok@bp}{\def\PY@tc##1{\textcolor[rgb]{0.00,0.50,0.00}{##1}}}
\@namedef{PY@tok@fm}{\def\PY@tc##1{\textcolor[rgb]{0.00,0.00,1.00}{##1}}}
\@namedef{PY@tok@vc}{\def\PY@tc##1{\textcolor[rgb]{0.10,0.09,0.49}{##1}}}
\@namedef{PY@tok@vg}{\def\PY@tc##1{\textcolor[rgb]{0.10,0.09,0.49}{##1}}}
\@namedef{PY@tok@vi}{\def\PY@tc##1{\textcolor[rgb]{0.10,0.09,0.49}{##1}}}
\@namedef{PY@tok@vm}{\def\PY@tc##1{\textcolor[rgb]{0.10,0.09,0.49}{##1}}}
\@namedef{PY@tok@sa}{\def\PY@tc##1{\textcolor[rgb]{0.73,0.13,0.13}{##1}}}
\@namedef{PY@tok@sb}{\def\PY@tc##1{\textcolor[rgb]{0.73,0.13,0.13}{##1}}}
\@namedef{PY@tok@sc}{\def\PY@tc##1{\textcolor[rgb]{0.73,0.13,0.13}{##1}}}
\@namedef{PY@tok@dl}{\def\PY@tc##1{\textcolor[rgb]{0.73,0.13,0.13}{##1}}}
\@namedef{PY@tok@s2}{\def\PY@tc##1{\textcolor[rgb]{0.73,0.13,0.13}{##1}}}
\@namedef{PY@tok@sh}{\def\PY@tc##1{\textcolor[rgb]{0.73,0.13,0.13}{##1}}}
\@namedef{PY@tok@s1}{\def\PY@tc##1{\textcolor[rgb]{0.73,0.13,0.13}{##1}}}
\@namedef{PY@tok@mb}{\def\PY@tc##1{\textcolor[rgb]{0.40,0.40,0.40}{##1}}}
\@namedef{PY@tok@mf}{\def\PY@tc##1{\textcolor[rgb]{0.40,0.40,0.40}{##1}}}
\@namedef{PY@tok@mh}{\def\PY@tc##1{\textcolor[rgb]{0.40,0.40,0.40}{##1}}}
\@namedef{PY@tok@mi}{\def\PY@tc##1{\textcolor[rgb]{0.40,0.40,0.40}{##1}}}
\@namedef{PY@tok@il}{\def\PY@tc##1{\textcolor[rgb]{0.40,0.40,0.40}{##1}}}
\@namedef{PY@tok@mo}{\def\PY@tc##1{\textcolor[rgb]{0.40,0.40,0.40}{##1}}}
\@namedef{PY@tok@ch}{\let\PY@it=\textit\def\PY@tc##1{\textcolor[rgb]{0.24,0.48,0.48}{##1}}}
\@namedef{PY@tok@cm}{\let\PY@it=\textit\def\PY@tc##1{\textcolor[rgb]{0.24,0.48,0.48}{##1}}}
\@namedef{PY@tok@cpf}{\let\PY@it=\textit\def\PY@tc##1{\textcolor[rgb]{0.24,0.48,0.48}{##1}}}
\@namedef{PY@tok@c1}{\let\PY@it=\textit\def\PY@tc##1{\textcolor[rgb]{0.24,0.48,0.48}{##1}}}
\@namedef{PY@tok@cs}{\let\PY@it=\textit\def\PY@tc##1{\textcolor[rgb]{0.24,0.48,0.48}{##1}}}

\def\PYZbs{\char`\\}
\def\PYZus{\char`\_}
\def\PYZob{\char`\{}
\def\PYZcb{\char`\}}
\def\PYZca{\char`\^}
\def\PYZam{\char`\&}
\def\PYZlt{\char`\<}
\def\PYZgt{\char`\>}
\def\PYZsh{\char`\#}
\def\PYZpc{\char`\%}
\def\PYZdl{\char`\$}
\def\PYZhy{\char`\-}
\def\PYZsq{\char`\'}
\def\PYZdq{\char`\"}
\def\PYZti{\char`\~}
% for compatibility with earlier versions
\def\PYZat{@}
\def\PYZlb{[}
\def\PYZrb{]}
\makeatother


    % For linebreaks inside Verbatim environment from package fancyvrb.
    \makeatletter
        \newbox\Wrappedcontinuationbox
        \newbox\Wrappedvisiblespacebox
        \newcommand*\Wrappedvisiblespace {\textcolor{red}{\textvisiblespace}}
        \newcommand*\Wrappedcontinuationsymbol {\textcolor{red}{\llap{\tiny$\m@th\hookrightarrow$}}}
        \newcommand*\Wrappedcontinuationindent {3ex }
        \newcommand*\Wrappedafterbreak {\kern\Wrappedcontinuationindent\copy\Wrappedcontinuationbox}
        % Take advantage of the already applied Pygments mark-up to insert
        % potential linebreaks for TeX processing.
        %        {, <, #, %, $, ' and ": go to next line.
        %        _, }, ^, &, >, - and ~: stay at end of broken line.
        % Use of \textquotesingle for straight quote.
        \newcommand*\Wrappedbreaksatspecials {%
            \def\PYGZus{\discretionary{\char`\_}{\Wrappedafterbreak}{\char`\_}}%
            \def\PYGZob{\discretionary{}{\Wrappedafterbreak\char`\{}{\char`\{}}%
            \def\PYGZcb{\discretionary{\char`\}}{\Wrappedafterbreak}{\char`\}}}%
            \def\PYGZca{\discretionary{\char`\^}{\Wrappedafterbreak}{\char`\^}}%
            \def\PYGZam{\discretionary{\char`\&}{\Wrappedafterbreak}{\char`\&}}%
            \def\PYGZlt{\discretionary{}{\Wrappedafterbreak\char`\<}{\char`\<}}%
            \def\PYGZgt{\discretionary{\char`\>}{\Wrappedafterbreak}{\char`\>}}%
            \def\PYGZsh{\discretionary{}{\Wrappedafterbreak\char`\#}{\char`\#}}%
            \def\PYGZpc{\discretionary{}{\Wrappedafterbreak\char`\%}{\char`\%}}%
            \def\PYGZdl{\discretionary{}{\Wrappedafterbreak\char`\$}{\char`\$}}%
            \def\PYGZhy{\discretionary{\char`\-}{\Wrappedafterbreak}{\char`\-}}%
            \def\PYGZsq{\discretionary{}{\Wrappedafterbreak\textquotesingle}{\textquotesingle}}%
            \def\PYGZdq{\discretionary{}{\Wrappedafterbreak\char`\"}{\char`\"}}%
            \def\PYGZti{\discretionary{\char`\~}{\Wrappedafterbreak}{\char`\~}}%
        }
        % Some characters . , ; ? ! / are not pygmentized.
        % This macro makes them "active" and they will insert potential linebreaks
        \newcommand*\Wrappedbreaksatpunct {%
            \lccode`\~`\.\lowercase{\def~}{\discretionary{\hbox{\char`\.}}{\Wrappedafterbreak}{\hbox{\char`\.}}}%
            \lccode`\~`\,\lowercase{\def~}{\discretionary{\hbox{\char`\,}}{\Wrappedafterbreak}{\hbox{\char`\,}}}%
            \lccode`\~`\;\lowercase{\def~}{\discretionary{\hbox{\char`\;}}{\Wrappedafterbreak}{\hbox{\char`\;}}}%
            \lccode`\~`\:\lowercase{\def~}{\discretionary{\hbox{\char`\:}}{\Wrappedafterbreak}{\hbox{\char`\:}}}%
            \lccode`\~`\?\lowercase{\def~}{\discretionary{\hbox{\char`\?}}{\Wrappedafterbreak}{\hbox{\char`\?}}}%
            \lccode`\~`\!\lowercase{\def~}{\discretionary{\hbox{\char`\!}}{\Wrappedafterbreak}{\hbox{\char`\!}}}%
            \lccode`\~`\/\lowercase{\def~}{\discretionary{\hbox{\char`\/}}{\Wrappedafterbreak}{\hbox{\char`\/}}}%
            \catcode`\.\active
            \catcode`\,\active
            \catcode`\;\active
            \catcode`\:\active
            \catcode`\?\active
            \catcode`\!\active
            \catcode`\/\active
            \lccode`\~`\~
        }
    \makeatother

    \let\OriginalVerbatim=\Verbatim
    \makeatletter
    \renewcommand{\Verbatim}[1][1]{%
        %\parskip\z@skip
        \sbox\Wrappedcontinuationbox {\Wrappedcontinuationsymbol}%
        \sbox\Wrappedvisiblespacebox {\FV@SetupFont\Wrappedvisiblespace}%
        \def\FancyVerbFormatLine ##1{\hsize\linewidth
            \vtop{\raggedright\hyphenpenalty\z@\exhyphenpenalty\z@
                \doublehyphendemerits\z@\finalhyphendemerits\z@
                \strut ##1\strut}%
        }%
        % If the linebreak is at a space, the latter will be displayed as visible
        % space at end of first line, and a continuation symbol starts next line.
        % Stretch/shrink are however usually zero for typewriter font.
        \def\FV@Space {%
            \nobreak\hskip\z@ plus\fontdimen3\font minus\fontdimen4\font
            \discretionary{\copy\Wrappedvisiblespacebox}{\Wrappedafterbreak}
            {\kern\fontdimen2\font}%
        }%

        % Allow breaks at special characters using \PYG... macros.
        \Wrappedbreaksatspecials
        % Breaks at punctuation characters . , ; ? ! and / need catcode=\active
        \OriginalVerbatim[#1,codes*=\Wrappedbreaksatpunct]%
    }
    \makeatother

    % Exact colors from NB
    \definecolor{incolor}{HTML}{303F9F}
    \definecolor{outcolor}{HTML}{D84315}
    \definecolor{cellborder}{HTML}{CFCFCF}
    \definecolor{cellbackground}{HTML}{F7F7F7}

    % prompt
    \makeatletter
    \newcommand{\boxspacing}{\kern\kvtcb@left@rule\kern\kvtcb@boxsep}
    \makeatother
    \newcommand{\prompt}[4]{
        {\ttfamily\llap{{\color{#2}[#3]:\hspace{3pt}#4}}\vspace{-\baselineskip}}
    }
    

    
    % Prevent overflowing lines due to hard-to-break entities
    \sloppy
    % Setup hyperref package
    \hypersetup{
      breaklinks=true,  % so long urls are correctly broken across lines
      colorlinks=true,
      urlcolor=urlcolor,
      linkcolor=linkcolor,
      citecolor=citecolor,
      }
    % Slightly bigger margins than the latex defaults
    
    \geometry{verbose,tmargin=1in,bmargin=1in,lmargin=1in,rmargin=1in}
    
    

\begin{document}
\section*{Autora: Mª Victoria Gómez Bifante}
\vspace{0.7cm}
\section*{\underline{HOLOGRAFÍA DIGITAL FUERA DE EJE}}


    \hypertarget{muxe9todo-de-fourier-o-tuxe9cnica-de-filtrado}{%
\section*{Método de Fourier o técnica de
filtrado}\label{muxe9todo-de-fourier-o-tuxe9cnica-de-filtrado}}

\vspace{0.7cm}

    \begin{tcolorbox}[breakable, size=fbox, boxrule=1pt, pad at break*=1mm,colback=cellbackground, colframe=cellborder]
\prompt{In}{incolor}{1}{\boxspacing}
\begin{Verbatim}[commandchars=\\\{\}]
\PY{k+kn}{import} \PY{n+nn}{cv2}
\PY{k+kn}{import} \PY{n+nn}{numpy} \PY{k}{as} \PY{n+nn}{np}
\PY{k+kn}{import} \PY{n+nn}{matplotlib}\PY{n+nn}{.}\PY{n+nn}{pyplot} \PY{k}{as} \PY{n+nn}{plt} 
\PY{k+kn}{from} \PY{n+nn}{matplotlib}\PY{n+nn}{.}\PY{n+nn}{colors} \PY{k+kn}{import} \PY{n}{LogNorm}
\PY{k+kn}{from} \PY{n+nn}{scipy} \PY{k+kn}{import} \PY{n}{cluster}\PY{p}{,} \PY{n}{constants}\PY{p}{,} \PY{n}{fftpack}\PY{p}{,} \PY{n}{integrate}
\end{Verbatim}
\end{tcolorbox}

    \hypertarget{creamos-las-gradillas-y-presentamos-los-paruxe1metros-fijos}{%
\section{Creamos las gradillas y presentamos los parámetros
fijos}\label{creamos-las-gradillas-y-presentamos-los-paruxe1metros-fijos}}

    \begin{tcolorbox}[breakable, size=fbox, boxrule=1pt, pad at break*=1mm,colback=cellbackground, colframe=cellborder]
\prompt{In}{incolor}{2}{\boxspacing}
\begin{Verbatim}[commandchars=\\\{\}]
\PY{n}{N} \PY{o}{=} \PY{l+m+mi}{1650}\PY{p}{;} \PY{c+c1}{\PYZsh{}número máximo de píxeles  recogidos de la imagen en la dirección x}
\PY{n}{M} \PY{o}{=} \PY{l+m+mi}{1160}\PY{p}{;} \PY{c+c1}{\PYZsh{}número máximo de píxeles  recogidos de la imagen en la dirección y}
\PY{n}{deltax}\PY{p}{,} \PY{n}{deltay} \PY{o}{=} \PY{p}{(}\PY{l+m+mf}{0.001}\PY{p}{,} \PY{l+m+mf}{0.001}\PY{p}{)}\PY{p}{;} \PY{c+c1}{\PYZsh{}distancia entre píxeles}
\PY{n}{X} \PY{o}{=} \PY{n}{np}\PY{o}{.}\PY{n}{linspace}\PY{p}{(}\PY{l+m+mi}{0}\PY{p}{,}\PY{l+m+mi}{1}\PY{p}{,}\PY{n}{num} \PY{o}{=} \PY{n}{N}\PY{o}{+}\PY{l+m+mi}{1}\PY{p}{)}\PY{p}{;}
\PY{n}{Y} \PY{o}{=} \PY{n}{np}\PY{o}{.}\PY{n}{linspace}\PY{p}{(}\PY{l+m+mi}{0}\PY{p}{,}\PY{l+m+mi}{1}\PY{p}{,}\PY{n}{num} \PY{o}{=} \PY{n}{M}\PY{o}{+}\PY{l+m+mi}{1}\PY{p}{)}\PY{p}{;}
\PY{n}{x}\PY{p}{,} \PY{n}{y} \PY{o}{=} \PY{n}{np}\PY{o}{.}\PY{n}{meshgrid}\PY{p}{(}\PY{n}{X}\PY{p}{,} \PY{n}{Y}\PY{p}{)}\PY{p}{;} \PY{c+c1}{\PYZsh{}creación del sistema N x M}
\PY{n}{theta} \PY{o}{=} \PY{n}{np}\PY{o}{.}\PY{n}{pi}\PY{o}{/}\PY{l+m+mi}{4}\PY{p}{;}  \PY{c+c1}{\PYZsh{}angulo interferencial 1 (rad)}
\PY{n}{lamb} \PY{o}{=} \PY{l+m+mf}{632.8} \PY{o}{*}\PY{l+m+mi}{10}\PY{o}{*}\PY{o}{*}\PY{p}{(}\PY{o}{\PYZhy{}}\PY{l+m+mi}{8}\PY{p}{)}\PY{p}{;} \PY{c+c1}{\PYZsh{}longitud de onda (632.8 nm)}
\PY{n}{z} \PY{o}{=}  \PY{n}{N} \PY{o}{*} \PY{n}{deltax}\PY{o}{*}\PY{o}{*}\PY{p}{(}\PY{l+m+mi}{2}\PY{p}{)} \PY{o}{/} \PY{n}{lamb} \PY{c+c1}{\PYZsh{}distancia de propagación hasta el holograma}
\PY{n}{k} \PY{o}{=} \PY{p}{(}\PY{l+m+mi}{2}\PY{o}{*}\PY{n}{np}\PY{o}{.}\PY{n}{pi}\PY{p}{)} \PY{o}{/}\PY{n}{lamb}\PY{p}{;} \PY{c+c1}{\PYZsh{}vector de propagacion en (rad/m)}
\end{Verbatim}
\end{tcolorbox}

    \hypertarget{definimos-las-funciones-que-vamos-a-utilizar}{%
\section{Definimos las funciones que vamos a
utilizar}\label{definimos-las-funciones-que-vamos-a-utilizar}}

    \begin{tcolorbox}[breakable, size=fbox, boxrule=1pt, pad at break*=1mm,colback=cellbackground, colframe=cellborder]
\prompt{In}{incolor}{3}{\boxspacing}
\begin{Verbatim}[commandchars=\\\{\}]
\PY{k}{def} \PY{n+nf}{UnitBox}\PY{p}{(}\PY{n}{coordenate1}\PY{p}{,} \PY{n}{coordenate2}\PY{p}{)}\PY{p}{:}
    \PY{n}{n1} \PY{o}{=} \PY{n}{coordenate1}\PY{o}{.}\PY{n}{ndim}\PY{p}{;}
    \PY{n}{n2} \PY{o}{=} \PY{n}{coordenate2}\PY{o}{.}\PY{n}{ndim}\PY{p}{;}
    \PY{n}{Radius}\PY{o}{=} \PY{n}{np}\PY{o}{.}\PY{n}{sqrt}\PY{p}{(}\PY{n}{coordenate1}\PY{o}{*}\PY{o}{*}\PY{l+m+mi}{2} \PY{o}{+} \PY{n}{coordenate2}\PY{o}{*}\PY{o}{*}\PY{l+m+mi}{2}\PY{p}{)}\PY{p}{;}
    \PY{n}{Disk} \PY{o}{=} \PY{n}{np}\PY{o}{.}\PY{n}{zeros\PYZus{}like}\PY{p}{(}\PY{p}{(}\PY{n}{Radius}\PY{p}{)}\PY{p}{)}\PY{p}{;}
    \PY{n}{limit} \PY{o}{=} \PY{l+m+mi}{1}\PY{o}{/}\PY{l+m+mi}{2}\PY{p}{;}
    \PY{k}{if} \PY{p}{(}\PY{n}{n1} \PY{o}{==} \PY{l+m+mi}{1} \PY{o}{\PYZam{}} \PY{n}{n2} \PY{o}{==} \PY{l+m+mi}{1}\PY{p}{)}\PY{p}{:}
        \PY{k}{for} \PY{n}{ii} \PY{o+ow}{in} \PY{n+nb}{range}\PY{p}{(}\PY{l+m+mi}{0}\PY{p}{,}\PY{n}{N}\PY{o}{\PYZhy{}}\PY{l+m+mi}{1}\PY{p}{)}\PY{p}{:}
            \PY{k}{if}  \PY{p}{(}\PY{p}{(}\PY{n}{coordenate1}\PY{p}{[}\PY{n}{ii}\PY{p}{]} \PY{o}{\PYZlt{}}\PY{o}{=} \PY{n}{limit}\PY{p}{)} \PY{o}{\PYZam{}} \PY{p}{(}\PY{n}{coordenate1}\PY{p}{[}\PY{n}{ii}\PY{p}{]} \PY{o}{\PYZgt{}}\PY{o}{=} \PY{o}{\PYZhy{}}\PY{n}{limit}\PY{p}{)}\PY{p}{)}\PY{p}{:}
                \PY{n}{Disk}\PY{p}{[}\PY{n}{ii}\PY{p}{]} \PY{o}{=} \PY{l+m+mf}{1.0}\PY{p}{;}
    \PY{k}{elif} \PY{p}{(}\PY{n}{n1} \PY{o}{==} \PY{l+m+mi}{2} \PY{o}{\PYZam{}} \PY{n}{n2} \PY{o}{==} \PY{l+m+mi}{2}\PY{p}{)}\PY{p}{:}
        \PY{k}{for} \PY{n}{ii} \PY{o+ow}{in} \PY{n+nb}{range}\PY{p}{(}\PY{l+m+mi}{0}\PY{p}{,}\PY{n}{N}\PY{o}{\PYZhy{}}\PY{l+m+mi}{1}\PY{p}{)}\PY{p}{:}
            \PY{k}{for} \PY{n}{jj} \PY{o+ow}{in} \PY{n+nb}{range}\PY{p}{(}\PY{l+m+mi}{0}\PY{p}{,}\PY{n}{N}\PY{o}{\PYZhy{}}\PY{l+m+mi}{1}\PY{p}{)}\PY{p}{:}
                \PY{k}{if}  \PY{p}{(}\PY{n}{Radius}\PY{p}{[}\PY{n}{ii}\PY{p}{]}\PY{p}{[}\PY{n}{jj}\PY{p}{]} \PY{o}{\PYZlt{}} \PY{n}{limit}\PY{p}{)}\PY{p}{:}
                    \PY{n}{Disk}\PY{p}{[}\PY{n}{ii}\PY{p}{]}\PY{p}{[}\PY{n}{jj}\PY{p}{]} \PY{o}{=} \PY{l+m+mf}{1.0}\PY{p}{;}
    \PY{k}{return} \PY{n}{Disk}

\PY{k}{def} \PY{n+nf}{filter\PYZus{}VanderLught}\PY{p}{(}\PY{n}{I\PYZus{}F\PYZus{}}\PY{p}{)}\PY{p}{:} \PY{c+c1}{\PYZsh{} Filtro de VanderLught}
    
    \PY{n}{N\PYZus{}2} \PY{o}{=} \PY{n+nb}{int}\PY{p}{(}\PY{n}{N}\PY{o}{/}\PY{l+m+mi}{2}\PY{p}{)}\PY{p}{;}
    \PY{n}{M\PYZus{}2} \PY{o}{=} \PY{n+nb}{int}\PY{p}{(}\PY{n}{M}\PY{o}{/}\PY{l+m+mi}{2}\PY{p}{)}\PY{p}{;}
    \PY{n}{Ux} \PY{o}{=} \PY{n}{np}\PY{o}{.}\PY{n}{arange}\PY{p}{(}\PY{o}{\PYZhy{}}\PY{n}{N\PYZus{}2}\PY{p}{,} \PY{n}{N\PYZus{}2}\PY{o}{+}\PY{l+m+mi}{1}\PY{p}{)}\PY{p}{;}
    \PY{n}{Uy} \PY{o}{=} \PY{n}{np}\PY{o}{.}\PY{n}{arange}\PY{p}{(}\PY{o}{\PYZhy{}}\PY{n}{M\PYZus{}2}\PY{p}{,} \PY{n}{M\PYZus{}2}\PY{o}{+}\PY{l+m+mi}{1}\PY{p}{)}\PY{p}{;}
    \PY{p}{[}\PY{n}{ux}\PY{p}{,} \PY{n}{uy}\PY{p}{]} \PY{o}{=} \PY{n}{np}\PY{o}{.}\PY{n}{meshgrid}\PY{p}{(}\PY{n}{Ux}\PY{p}{,} \PY{n}{Uy}\PY{p}{)}\PY{p}{;}
    \PY{n}{f} \PY{o}{=} \PY{l+m+mi}{100}\PY{p}{;}
    \PY{n}{filtered\PYZus{}signal} \PY{o}{=} \PY{p}{(}\PY{n}{ux} \PY{o}{\PYZgt{}} \PY{p}{(}\PY{n}{f} \PY{o}{\PYZhy{}} \PY{n}{f}\PY{o}{/}\PY{o}{/}\PY{l+m+mi}{2} \PY{p}{)}\PY{p}{)}\PY{o}{*}\PY{p}{(}\PY{n}{ux} \PY{o}{\PYZlt{}} \PY{p}{(}\PY{n}{f}  \PY{o}{+} \PY{n}{f}\PY{o}{/}\PY{o}{/}\PY{l+m+mi}{2}\PY{p}{)}\PY{p}{)}\PY{p}{;}

    \PY{k}{return} \PY{n}{filtered\PYZus{}signal}
\end{Verbatim}
\end{tcolorbox}

    \hypertarget{extraemos-la-informaciuxf3n-de-la-imagen}{%
\section{Extraemos la información de la
imagen}\label{extraemos-la-informaciuxf3n-de-la-imagen}}

    \begin{tcolorbox}[breakable, size=fbox, boxrule=1pt, pad at break*=1mm,colback=cellbackground, colframe=cellborder]
\prompt{In}{incolor}{4}{\boxspacing}
\begin{Verbatim}[commandchars=\\\{\}]
\PY{n}{img} \PY{o}{=} \PY{n}{cv2}\PY{o}{.}\PY{n}{imread}\PY{p}{(}\PY{l+s+s1}{\PYZsq{}}\PY{l+s+s1}{holography\PYZus{}test\PYZus{}image\PYZus{}rgb.jpeg}\PY{l+s+s1}{\PYZsq{}}\PY{p}{,} \PY{n}{cv2}\PY{o}{.}\PY{n}{IMREAD\PYZus{}UNCHANGED}\PY{p}{)}\PY{p}{;} \PY{c+c1}{\PYZsh{} equivale a la matriz de variación de amplitud de la onda objeto}
\PY{n}{img\PYZus{}rgb} \PY{o}{=} \PY{n}{np}\PY{o}{.}\PY{n}{float64}\PY{p}{(}\PY{n}{img}\PY{p}{)}\PY{p}{;} \PY{c+c1}{\PYZsh{}añadimos precisión}
\PY{n+nb}{print}\PY{p}{(}\PY{l+s+s1}{\PYZsq{}}\PY{l+s+s1}{Dimensiones originales de la imagen: }\PY{l+s+s1}{\PYZsq{}}\PY{p}{,}\PY{n}{img}\PY{o}{.}\PY{n}{shape}\PY{p}{)}\PY{p}{;}
\PY{n+nb}{print}\PY{p}{(}\PY{l+s+s1}{\PYZsq{}}\PY{l+s+s1}{Número total de dimensiones: }\PY{l+s+s1}{\PYZsq{}}\PY{p}{,}\PY{n}{img}\PY{o}{.}\PY{n}{ndim}\PY{p}{)}\PY{p}{;}
\PY{n}{plt}\PY{o}{.}\PY{n}{imshow}\PY{p}{(}\PY{n}{img}\PY{p}{,} \PY{n}{extent}\PY{o}{=}\PY{p}{[}\PY{o}{\PYZhy{}}\PY{n}{N}\PY{p}{,} \PY{n}{N}\PY{p}{,} \PY{o}{\PYZhy{}}\PY{n}{M}\PY{p}{,} \PY{n}{M}\PY{p}{]}\PY{p}{)}
\PY{n}{plt}\PY{o}{.}\PY{n}{title}\PY{p}{(}\PY{l+s+s2}{\PYZdq{}}\PY{l+s+s2}{Imagen Original}\PY{l+s+s2}{\PYZdq{}}\PY{p}{)}
\PY{n}{plt}\PY{o}{.}\PY{n}{xlabel}\PY{p}{(}\PY{l+s+s2}{\PYZdq{}}\PY{l+s+s2}{píxeles recogidos en la coordenada x}\PY{l+s+s2}{\PYZdq{}}\PY{p}{)}
\PY{n}{plt}\PY{o}{.}\PY{n}{ylabel}\PY{p}{(}\PY{l+s+s2}{\PYZdq{}}\PY{l+s+s2}{píxeles recogidos en la coordenada y}\PY{l+s+s2}{\PYZdq{}}\PY{p}{)}
\end{Verbatim}
\end{tcolorbox}

    \begin{Verbatim}[commandchars=\\\{\}]
Dimensiones originales de la imagen:  (1161, 1651, 3)
Número total de dimensiones:  3
    \end{Verbatim}

            \begin{tcolorbox}[breakable, size=fbox, boxrule=.5pt, pad at break*=1mm, opacityfill=0]
\prompt{Out}{outcolor}{4}{\boxspacing}
\begin{Verbatim}[commandchars=\\\{\}]
Text(0, 0.5, 'píxeles recogidos en la coordenada y')
\end{Verbatim}
\end{tcolorbox}
        
    \begin{center}
    \adjustimage{max size={  0.8\linewidth}{  0.8\paperheight}}{output_8_2.png}
    \end{center}
    
    \hypertarget{la-transformamos-a-blanco-y-negro}{%
\section{La transformamos a blanco y
negro}\label{la-transformamos-a-blanco-y-negro}}

    \begin{tcolorbox}[breakable, size=fbox, boxrule=1pt, pad at break*=1mm,colback=cellbackground, colframe=cellborder]
\prompt{In}{incolor}{5}{\boxspacing}
\begin{Verbatim}[commandchars=\\\{\}]
\PY{n}{img} \PY{o}{=} \PY{n}{cv2}\PY{o}{.}\PY{n}{resize}\PY{p}{(}\PY{n}{img}\PY{p}{,} \PY{p}{(}\PY{n}{N}\PY{p}{,} \PY{n}{M}\PY{p}{)}\PY{p}{)}\PY{p}{;}
\PY{n}{img\PYZus{}g} \PY{o}{=} \PY{n}{cv2}\PY{o}{.}\PY{n}{imread}\PY{p}{(}\PY{l+s+s1}{\PYZsq{}}\PY{l+s+s1}{holography\PYZus{}test\PYZus{}image\PYZus{}rgb.jpeg}\PY{l+s+s1}{\PYZsq{}}\PY{p}{,} \PY{n}{cv2}\PY{o}{.}\PY{n}{IMREAD\PYZus{}GRAYSCALE}\PY{p}{)}\PY{p}{;} \PY{c+c1}{\PYZsh{} miminizamos la información de la amplitud }
\PY{n}{img\PYZus{}gray} \PY{o}{=} \PY{n}{np}\PY{o}{.}\PY{n}{float64}\PY{p}{(}\PY{n}{img\PYZus{}g}\PY{p}{)}\PY{p}{;} \PY{c+c1}{\PYZsh{}añadimos precisión}
\PY{n+nb}{print}\PY{p}{(}\PY{l+s+s1}{\PYZsq{}}\PY{l+s+s1}{Dimensiones de la imagen en blanco y negro: }\PY{l+s+s1}{\PYZsq{}}\PY{p}{,}\PY{n}{img\PYZus{}gray}\PY{o}{.}\PY{n}{shape}\PY{p}{)}\PY{p}{;}
\PY{n+nb}{print}\PY{p}{(}\PY{l+s+s1}{\PYZsq{}}\PY{l+s+s1}{Número total de dimensiones de la imagen en blanco y negro: }\PY{l+s+s1}{\PYZsq{}}\PY{p}{,}\PY{n}{img\PYZus{}gray}\PY{o}{.}\PY{n}{ndim}\PY{p}{)}\PY{p}{;}
\PY{n}{A\PYZus{}0}\PY{o}{=} \PY{n}{np}\PY{o}{.}\PY{n}{fft}\PY{o}{.}\PY{n}{fftshift}\PY{p}{(}\PY{n}{np}\PY{o}{.}\PY{n}{fft}\PY{o}{.}\PY{n}{fft2}\PY{p}{(}\PY{n}{img\PYZus{}gray}\PY{p}{)}\PY{p}{)}\PY{p}{;} \PY{c+c1}{\PYZsh{}amplitud distribución de amplitud del haz objeto}
\PY{n}{plt}\PY{o}{.}\PY{n}{imshow}\PY{p}{(}\PY{n}{img\PYZus{}g}\PY{p}{,} \PY{n}{cmap} \PY{o}{=} \PY{l+s+s1}{\PYZsq{}}\PY{l+s+s1}{gray}\PY{l+s+s1}{\PYZsq{}}\PY{p}{,} \PY{n}{extent}\PY{o}{=}\PY{p}{[}\PY{o}{\PYZhy{}}\PY{n}{N}\PY{p}{,} \PY{n}{N}\PY{p}{,} \PY{o}{\PYZhy{}}\PY{n}{M}\PY{p}{,} \PY{n}{M}\PY{p}{]}\PY{p}{)}
\PY{n}{plt}\PY{o}{.}\PY{n}{title}\PY{p}{(}\PY{l+s+s2}{\PYZdq{}}\PY{l+s+s2}{Imagen en la escala de grises}\PY{l+s+s2}{\PYZdq{}}\PY{p}{)}
\PY{n}{plt}\PY{o}{.}\PY{n}{xlabel}\PY{p}{(}\PY{l+s+s2}{\PYZdq{}}\PY{l+s+s2}{píxeles recogidos en la coordenada x}\PY{l+s+s2}{\PYZdq{}}\PY{p}{)}
\PY{n}{plt}\PY{o}{.}\PY{n}{ylabel}\PY{p}{(}\PY{l+s+s2}{\PYZdq{}}\PY{l+s+s2}{píxeles recogidos en la coordenada y}\PY{l+s+s2}{\PYZdq{}}\PY{p}{)}
\end{Verbatim}
\end{tcolorbox}

    \begin{Verbatim}[commandchars=\\\{\}]
Dimensiones de la imagen en blanco y negro:  (1161, 1651)
Número total de dimensiones de la imagen en blanco y negro:  2
    \end{Verbatim}

            \begin{tcolorbox}[breakable, size=fbox, boxrule=.5pt, pad at break*=1mm, opacityfill=0]
\prompt{Out}{outcolor}{5}{\boxspacing}
\begin{Verbatim}[commandchars=\\\{\}]
Text(0, 0.5, 'píxeles recogidos en la coordenada y')
\end{Verbatim}
\end{tcolorbox}
        
    \begin{center}
    \adjustimage{max size={  0.8\linewidth}{  0.8\paperheight}}{output_10_2.png}
    \end{center}
 
    
    \hypertarget{intensidad-de-la-interferencia-en-el-dominio-espacial}{%
\section{Intensidad de la interferencia en el dominio
espacial}\label{intensidad-de-la-interferencia-en-el-dominio-espacial}}

    \begin{tcolorbox}[breakable, size=fbox, boxrule=1pt, pad at break*=1mm,colback=cellbackground, colframe=cellborder]
\prompt{In}{incolor}{6}{\boxspacing}
\begin{Verbatim}[commandchars=\\\{\}]
\PY{n}{UO} \PY{o}{=} \PY{n}{img\PYZus{}gray} \PY{o}{*} \PY{n}{np}\PY{o}{.}\PY{n}{exp}\PY{p}{(}\PY{l+m+mi}{1}\PY{n}{j}\PY{o}{*}\PY{n}{np}\PY{o}{.}\PY{n}{pi}\PY{o}{/}\PY{l+m+mi}{2}\PY{o}{*}\PY{n}{UnitBox}\PY{p}{(}\PY{n}{X}\PY{p}{,}\PY{n}{X}\PY{p}{)}\PY{p}{)}\PY{o}{*}\PY{n}{np}\PY{o}{.}\PY{n}{exp}\PY{p}{(}\PY{l+m+mi}{1}\PY{n}{j}\PY{o}{*}\PY{n}{k}\PY{o}{*}\PY{n}{x}\PY{o}{*}\PY{n}{np}\PY{o}{.}\PY{n}{sin}\PY{p}{(}\PY{n}{theta}\PY{p}{)}\PY{p}{)}\PY{p}{;} \PY{c+c1}{\PYZsh{}haz objeto}
\PY{n}{UO}\PY{o}{=} \PY{n}{UO}\PY{o}{/} \PY{n}{np}\PY{o}{.}\PY{n}{max}\PY{p}{(}\PY{n}{UO}\PY{p}{)}\PY{p}{;} \PY{c+c1}{\PYZsh{}normalizamos}
\PY{n}{UO\PYZus{}abs} \PY{o}{=} \PY{n}{np}\PY{o}{.}\PY{n}{abs}\PY{p}{(}\PY{n}{UO}\PY{p}{)}\PY{o}{*}\PY{o}{*}\PY{l+m+mi}{2}\PY{p}{;} \PY{c+c1}{\PYZsh{}devuelve valores absolutos de la matriz UO}
\PY{n}{UO\PYZus{}abs} \PY{o}{=} \PY{n}{np}\PY{o}{.}\PY{n}{where}\PY{p}{(}\PY{n}{UO\PYZus{}abs} \PY{o}{\PYZgt{}} \PY{l+m+mf}{0.0000000001}\PY{p}{,} \PY{n}{UO\PYZus{}abs}\PY{p}{,} \PY{o}{\PYZhy{}}\PY{l+m+mi}{10}\PY{p}{)}\PY{p}{;} \PY{c+c1}{\PYZsh{}transformamos UO a escala logarítmica}
\PY{n}{UO\PYZus{}log} \PY{o}{=} \PY{n}{np}\PY{o}{.}\PY{n}{log10}\PY{p}{(}\PY{n}{UO\PYZus{}abs}\PY{p}{,} \PY{n}{out}\PY{o}{=}\PY{n}{UO\PYZus{}abs}\PY{p}{,} \PY{n}{where}\PY{o}{=}\PY{n}{UO\PYZus{}abs} \PY{o}{\PYZgt{}} \PY{l+m+mi}{0}\PY{p}{)}\PY{p}{;}      
\PY{n}{UR} \PY{o}{=} \PY{n}{np}\PY{o}{.}\PY{n}{exp}\PY{p}{(}\PY{l+m+mi}{1}\PY{n}{j}\PY{o}{*}\PY{n}{k}\PY{o}{*}\PY{n}{x}\PY{p}{)}\PY{p}{;} \PY{c+c1}{\PYZsh{}haz de referencia}
\PY{n}{UR}\PY{o}{=} \PY{n}{UR}\PY{o}{/} \PY{n}{np}\PY{o}{.}\PY{n}{max}\PY{p}{(}\PY{n}{UR}\PY{p}{)}\PY{p}{;} \PY{c+c1}{\PYZsh{}normalizamos}
\end{Verbatim}
\end{tcolorbox}

    \begin{tcolorbox}[breakable, size=fbox, boxrule=1pt, pad at break*=1mm,colback=cellbackground, colframe=cellborder]
\prompt{In}{incolor}{7}{\boxspacing}
\begin{Verbatim}[commandchars=\\\{\}]
\PY{n}{I} \PY{o}{=} \PY{n}{np}\PY{o}{.}\PY{n}{abs}\PY{p}{(}\PY{n}{UO} \PY{o}{+} \PY{n}{UR}\PY{p}{)}\PY{o}{*}\PY{o}{*}\PY{l+m+mi}{2}\PY{p}{;} \PY{c+c1}{\PYZsh{}intensidad resultante de la interferencia entre UO y UR}
\end{Verbatim}
\end{tcolorbox}

    \hypertarget{intensidad-de-la-interferencia-en-el-dominio-frecuencial}{%
\section{Intensidad de la interferencia en el dominio
frecuencial}\label{intensidad-de-la-interferencia-en-el-dominio-frecuencial}}

    \begin{tcolorbox}[breakable, size=fbox, boxrule=1pt, pad at break*=1mm,colback=cellbackground, colframe=cellborder]
\prompt{In}{incolor}{8}{\boxspacing}
\begin{Verbatim}[commandchars=\\\{\}]
\PY{n}{I\PYZus{}F}\PY{o}{=} \PY{n}{np}\PY{o}{.}\PY{n}{fft}\PY{o}{.}\PY{n}{fftshift}\PY{p}{(}\PY{n}{np}\PY{o}{.}\PY{n}{fft}\PY{o}{.}\PY{n}{fft2}\PY{p}{(}\PY{n}{np}\PY{o}{.}\PY{n}{fft}\PY{o}{.}\PY{n}{fftshift}\PY{p}{(}\PY{n}{I}\PY{p}{)}\PY{p}{)}\PY{p}{)} \PY{c+c1}{\PYZsh{}distribución de amplitud del haz objeto inicial}
\PY{n}{I\PYZus{}F\PYZus{}abs} \PY{o}{=} \PY{n}{np}\PY{o}{.}\PY{n}{abs}\PY{p}{(}\PY{n}{I\PYZus{}F}\PY{p}{)}\PY{p}{;} \PY{c+c1}{\PYZsh{} calculamos el valor absoluto de la intensidad en el plano de Fourier}
\PY{n}{I\PYZus{}F\PYZus{}log} \PY{o}{=} \PY{l+m+mi}{20} \PY{o}{*} \PY{n}{np}\PY{o}{.}\PY{n}{log10}\PY{p}{(}\PY{n}{I\PYZus{}F\PYZus{}abs}\PY{p}{)}\PY{p}{;} \PY{c+c1}{\PYZsh{}lo convertimos a la escala logratimica}
\end{Verbatim}
\end{tcolorbox}

    \hypertarget{aplicamos-el-filtro-sobre-la-intensidad-espectral}{%
\section{Aplicamos el filtro sobre la intensidad
espectral}\label{aplicamos-el-filtro-sobre-la-intensidad-espectral}}

    \begin{tcolorbox}[breakable, size=fbox, boxrule=1pt, pad at break*=1mm,colback=cellbackground, colframe=cellborder]
\prompt{In}{incolor}{9}{\boxspacing}
\begin{Verbatim}[commandchars=\\\{\}]
\PY{n}{fil}\PY{o}{=} \PY{n}{filter\PYZus{}VanderLught}\PY{p}{(}\PY{n}{I\PYZus{}F}\PY{p}{)}
\PY{n}{G} \PY{o}{=} \PY{n}{I\PYZus{}F} \PY{o}{*}\PY{n}{fil}\PY{p}{;} \PY{c+c1}{\PYZsh{} G es la distribución de amplitud resultante tras aplicar el filtro}
\PY{n}{G\PYZus{}abs} \PY{o}{=} \PY{n}{np}\PY{o}{.}\PY{n}{abs}\PY{p}{(}\PY{n}{G}\PY{p}{)} \PY{c+c1}{\PYZsh{}calculamos los valores absolutos de la matriz G}
\PY{n}{G\PYZus{}abs\PYZus{}} \PY{o}{=} \PY{n}{np}\PY{o}{.}\PY{n}{where}\PY{p}{(}\PY{n}{G\PYZus{}abs} \PY{o}{\PYZgt{}} \PY{l+m+mf}{0.0000000001}\PY{p}{,} \PY{n}{G\PYZus{}abs}\PY{p}{,} \PY{o}{\PYZhy{}}\PY{l+m+mi}{10}\PY{p}{)}\PY{p}{;}
\PY{n}{G\PYZus{}log} \PY{o}{=} \PY{n}{np}\PY{o}{.}\PY{n}{log10}\PY{p}{(}\PY{n}{G\PYZus{}abs}\PY{p}{,} \PY{n}{out}\PY{o}{=}\PY{n}{G\PYZus{}abs}\PY{p}{,} \PY{n}{where}\PY{o}{=}\PY{n}{G\PYZus{}abs} \PY{o}{\PYZgt{}} \PY{l+m+mi}{0}\PY{p}{)}\PY{p}{;} \PY{c+c1}{\PYZsh{}transformamos G a escala logarítmica}
\end{Verbatim}
\end{tcolorbox}

    \hypertarget{reconstrucciuxf3n-de-la-onda-objeto}{%
\section{Reconstrucción de la onda
objeto}\label{reconstrucciuxf3n-de-la-onda-objeto}}

    \begin{tcolorbox}[breakable, size=fbox, boxrule=1pt, pad at break*=1mm,colback=cellbackground, colframe=cellborder]
\prompt{In}{incolor}{10}{\boxspacing}
\begin{Verbatim}[commandchars=\\\{\}]
\PY{n}{G} \PY{o}{=} \PY{n}{np}\PY{o}{.}\PY{n}{fft}\PY{o}{.}\PY{n}{ifftshift}\PY{p}{(}\PY{n}{np}\PY{o}{.}\PY{n}{fft}\PY{o}{.}\PY{n}{ifft2}\PY{p}{(}\PY{n}{np}\PY{o}{.}\PY{n}{fft}\PY{o}{.}\PY{n}{ifftshift}\PY{p}{(}\PY{n}{G} \PY{p}{)}\PY{p}{)}\PY{p}{)}
\PY{n}{CF} \PY{o}{=} \PY{n}{UR} \PY{o}{*} \PY{n}{G}\PY{p}{;}
\PY{n}{CF\PYZus{}abs} \PY{o}{=} \PY{n}{np}\PY{o}{.}\PY{n}{abs}\PY{p}{(}\PY{n}{CF}\PY{p}{)} \PY{c+c1}{\PYZsh{}devuelve valores absolutos de la matriz CF }
\PY{n}{CF\PYZus{}abs\PYZus{}} \PY{o}{=} \PY{n}{np}\PY{o}{.}\PY{n}{where}\PY{p}{(}\PY{n}{CF\PYZus{}abs} \PY{o}{\PYZgt{}} \PY{l+m+mf}{0.0000000001}\PY{p}{,} \PY{n}{CF\PYZus{}abs}\PY{p}{,} \PY{o}{\PYZhy{}}\PY{l+m+mi}{10}\PY{p}{)}\PY{p}{;}
\PY{n}{CF\PYZus{}log} \PY{o}{=} \PY{n}{np}\PY{o}{.}\PY{n}{log10}\PY{p}{(}\PY{n}{CF\PYZus{}abs}\PY{p}{,} \PY{n}{out}\PY{o}{=}\PY{n}{CF\PYZus{}abs}\PY{p}{,} \PY{n}{where}\PY{o}{=}\PY{n}{CF\PYZus{}abs} \PY{o}{\PYZgt{}} \PY{l+m+mi}{0}\PY{p}{)}\PY{p}{;} \PY{c+c1}{\PYZsh{}transformamos CF a escala logarítmica para tener información más precisa}
\end{Verbatim}
\end{tcolorbox}

    \hypertarget{representaciuxf3n-de-los-resultados}{%
\section{Representación de los
resultados}\label{representaciuxf3n-de-los-resultados}}

    \begin{tcolorbox}[breakable, size=fbox, boxrule=1pt, pad at break*=1mm,colback=cellbackground, colframe=cellborder]
\prompt{In}{incolor}{11}{\boxspacing}
\begin{Verbatim}[commandchars=\\\{\}]
\PY{n}{plt}\PY{o}{.}\PY{n}{figure}\PY{p}{(}\PY{p}{)}
\PY{n}{plt}\PY{o}{.}\PY{n}{subplot}\PY{p}{(}\PY{p}{)}
\PY{n}{plt}\PY{o}{.}\PY{n}{imshow}\PY{p}{(}\PY{n}{UO\PYZus{}log}\PY{p}{,} \PY{n}{extent}\PY{o}{=}\PY{p}{[}\PY{o}{\PYZhy{}}\PY{n}{N}\PY{p}{,} \PY{n}{N}\PY{p}{,} \PY{o}{\PYZhy{}}\PY{n}{M}\PY{p}{,} \PY{n}{M}\PY{p}{]}\PY{p}{)}
\PY{n}{plt}\PY{o}{.}\PY{n}{title}\PY{p}{(}\PY{l+s+s1}{\PYZsq{}}\PY{l+s+s1}{Amplitud de la onda objeto}\PY{l+s+s1}{\PYZsq{}}\PY{p}{)}
\PY{n}{plt}\PY{o}{.}\PY{n}{xlabel}\PY{p}{(}\PY{l+s+s2}{\PYZdq{}}\PY{l+s+s2}{Coordenada normalizada x}\PY{l+s+s2}{\PYZdq{}}\PY{p}{)}
\PY{n}{plt}\PY{o}{.}\PY{n}{ylabel}\PY{p}{(}\PY{l+s+s2}{\PYZdq{}}\PY{l+s+s2}{Coordenada normalizada y}\PY{l+s+s2}{\PYZdq{}}\PY{p}{)}

\PY{n}{plt}\PY{o}{.}\PY{n}{figure}\PY{p}{(}\PY{p}{)}
\PY{n}{plt}\PY{o}{.}\PY{n}{subplot}\PY{p}{(}\PY{p}{)}
\PY{n}{plt}\PY{o}{.}\PY{n}{imshow}\PY{p}{(}\PY{n}{np}\PY{o}{.}\PY{n}{angle}\PY{p}{(}\PY{n}{UO}\PY{p}{)}\PY{p}{,} \PY{n}{extent}\PY{o}{=}\PY{p}{[}\PY{o}{\PYZhy{}}\PY{n}{N}\PY{p}{,} \PY{n}{N}\PY{p}{,} \PY{o}{\PYZhy{}}\PY{n}{M}\PY{p}{,} \PY{n}{M}\PY{p}{]}\PY{p}{)}
\PY{n}{plt}\PY{o}{.}\PY{n}{title}\PY{p}{(}\PY{l+s+s1}{\PYZsq{}}\PY{l+s+s1}{Fase de la onda objeto.}\PY{l+s+s1}{\PYZsq{}}\PY{p}{)}
\PY{n}{plt}\PY{o}{.}\PY{n}{xlabel}\PY{p}{(}\PY{l+s+s2}{\PYZdq{}}\PY{l+s+s2}{Coordenada normalizadav x}\PY{l+s+s2}{\PYZdq{}}\PY{p}{)}
\PY{n}{plt}\PY{o}{.}\PY{n}{ylabel}\PY{p}{(}\PY{l+s+s2}{\PYZdq{}}\PY{l+s+s2}{Coordenada normalizada y}\PY{l+s+s2}{\PYZdq{}}\PY{p}{)}

\PY{n}{plt}\PY{o}{.}\PY{n}{figure}\PY{p}{(}\PY{p}{)}
\PY{n}{plt}\PY{o}{.}\PY{n}{subplot}\PY{p}{(}\PY{p}{)}
\PY{n}{plt}\PY{o}{.}\PY{n}{imshow}\PY{p}{(}\PY{n}{np}\PY{o}{.}\PY{n}{abs}\PY{p}{(}\PY{n}{UR}\PY{p}{)}\PY{p}{,} \PY{n}{extent}\PY{o}{=}\PY{p}{[}\PY{o}{\PYZhy{}}\PY{n}{N}\PY{p}{,} \PY{n}{N}\PY{p}{,} \PY{o}{\PYZhy{}}\PY{n}{M}\PY{p}{,} \PY{n}{M}\PY{p}{]}\PY{p}{)}
\PY{n}{plt}\PY{o}{.}\PY{n}{title}\PY{p}{(}\PY{l+s+s1}{\PYZsq{}}\PY{l+s+s1}{Amplitud de la onda de referencia.}\PY{l+s+s1}{\PYZsq{}}\PY{p}{)}
\PY{n}{plt}\PY{o}{.}\PY{n}{xlabel}\PY{p}{(}\PY{l+s+s2}{\PYZdq{}}\PY{l+s+s2}{Coordenada normalizada x}\PY{l+s+s2}{\PYZdq{}}\PY{p}{)}
\PY{n}{plt}\PY{o}{.}\PY{n}{ylabel}\PY{p}{(}\PY{l+s+s2}{\PYZdq{}}\PY{l+s+s2}{Coordenada normalizada y}\PY{l+s+s2}{\PYZdq{}}\PY{p}{)}

\PY{n}{plt}\PY{o}{.}\PY{n}{figure}\PY{p}{(}\PY{p}{)}
\PY{n}{plt}\PY{o}{.}\PY{n}{subplot}\PY{p}{(}\PY{p}{)}
\PY{n}{plt}\PY{o}{.}\PY{n}{imshow}\PY{p}{(}\PY{n}{np}\PY{o}{.}\PY{n}{angle}\PY{p}{(}\PY{n}{UR}\PY{p}{)}\PY{p}{,} \PY{n}{extent}\PY{o}{=}\PY{p}{[}\PY{o}{\PYZhy{}}\PY{n}{N}\PY{p}{,} \PY{n}{N}\PY{p}{,} \PY{o}{\PYZhy{}}\PY{n}{M}\PY{p}{,} \PY{n}{M}\PY{p}{]}\PY{p}{)}
\PY{n}{plt}\PY{o}{.}\PY{n}{title}\PY{p}{(}\PY{l+s+s1}{\PYZsq{}}\PY{l+s+s1}{Fase de la onda de referencia.}\PY{l+s+s1}{\PYZsq{}}\PY{p}{)}
\PY{n}{plt}\PY{o}{.}\PY{n}{xlabel}\PY{p}{(}\PY{l+s+s2}{\PYZdq{}}\PY{l+s+s2}{Coordenada normalizada x}\PY{l+s+s2}{\PYZdq{}}\PY{p}{)}
\PY{n}{plt}\PY{o}{.}\PY{n}{ylabel}\PY{p}{(}\PY{l+s+s2}{\PYZdq{}}\PY{l+s+s2}{Coordenada normalizada y}\PY{l+s+s2}{\PYZdq{}}\PY{p}{)}

\PY{n}{plt}\PY{o}{.}\PY{n}{figure}\PY{p}{(}\PY{p}{)}
\PY{n}{plt}\PY{o}{.}\PY{n}{subplot}\PY{p}{(}\PY{p}{)}
\PY{n}{plt}\PY{o}{.}\PY{n}{imshow}\PY{p}{(}\PY{n}{I}\PY{p}{,} \PY{n}{extent}\PY{o}{=}\PY{p}{[}\PY{o}{\PYZhy{}}\PY{n}{N}\PY{p}{,} \PY{n}{N}\PY{p}{,} \PY{o}{\PYZhy{}}\PY{n}{M}\PY{p}{,} \PY{n}{M}\PY{p}{]}\PY{p}{)}
\PY{n}{plt}\PY{o}{.}\PY{n}{title}\PY{p}{(}\PY{l+s+s1}{\PYZsq{}}\PY{l+s+s1}{Holograma resultante de la interferencia}\PY{l+s+s1}{\PYZsq{}}\PY{p}{)}
\PY{n}{plt}\PY{o}{.}\PY{n}{xlabel}\PY{p}{(}\PY{l+s+s2}{\PYZdq{}}\PY{l+s+s2}{Coordenada normalizada x}\PY{l+s+s2}{\PYZdq{}}\PY{p}{)}
\PY{n}{plt}\PY{o}{.}\PY{n}{ylabel}\PY{p}{(}\PY{l+s+s2}{\PYZdq{}}\PY{l+s+s2}{Coordenada normalizada y}\PY{l+s+s2}{\PYZdq{}}\PY{p}{)}
\end{Verbatim}
\end{tcolorbox}

            \begin{tcolorbox}[breakable, size=fbox, boxrule=.5pt, pad at break*=1mm, opacityfill=0]
\prompt{Out}{outcolor}{11}{\boxspacing}
\begin{Verbatim}[commandchars=\\\{\}]
Text(0, 0.5, 'Coordenada normalizada y')
\end{Verbatim}
\end{tcolorbox}
        
    \begin{center}
    \adjustimage{max size={  0.8\linewidth}{  0.8\paperheight}}{output_21_1.png}
    \end{center}
        \vspace{0.7cm}
    \begin{center}
    \adjustimage{max size={  0.8\linewidth}{  0.8\paperheight}}{output_21_2.png}
    \end{center}
     \vspace{0.7cm}
    
    \begin{center}
    \adjustimage{max size={  0.8\linewidth}{  0.8\paperheight}}{output_21_3.png}
    \end{center}
   
        \vspace{0.7cm}
    \begin{center}
    \adjustimage{max size={  0.8\linewidth}{  0.8\paperheight}}{output_21_4.png}
    \end{center}
 
        \vspace{0.7cm}
    \begin{center}
    \adjustimage{max size={  0.8\linewidth}{  0.8\paperheight}}{output_21_5.png}
    \end{center}
     \vspace{0.7cm}
    
    \begin{tcolorbox}[breakable, size=fbox, boxrule=1pt, pad at break*=1mm,colback=cellbackground, colframe=cellborder]
\prompt{In}{incolor}{12}{\boxspacing}
\begin{Verbatim}[commandchars=\\\{\}]
\PY{n}{plt}\PY{o}{.}\PY{n}{figure}\PY{p}{(}\PY{p}{)}
\PY{n}{plt}\PY{o}{.}\PY{n}{imshow}\PY{p}{(}\PY{n}{I\PYZus{}F\PYZus{}log}\PY{p}{,} \PY{n}{extent}\PY{o}{=}\PY{p}{[}\PY{o}{\PYZhy{}}\PY{n}{N}\PY{p}{,} \PY{n}{N}\PY{p}{,} \PY{o}{\PYZhy{}}\PY{n}{M}\PY{p}{,} \PY{n}{M}\PY{p}{]}\PY{p}{)}
\PY{n}{plt}\PY{o}{.}\PY{n}{title}\PY{p}{(}\PY{l+s+s2}{\PYZdq{}}\PY{l+s+s2}{Holograma en el plano de Fourier}\PY{l+s+s2}{\PYZdq{}}\PY{p}{)}
\PY{n}{plt}\PY{o}{.}\PY{n}{xlabel}\PY{p}{(}\PY{l+s+s2}{\PYZdq{}}\PY{l+s+s2}{Coordenada normalizada x}\PY{l+s+s2}{\PYZdq{}}\PY{p}{)}
\PY{n}{plt}\PY{o}{.}\PY{n}{ylabel}\PY{p}{(}\PY{l+s+s2}{\PYZdq{}}\PY{l+s+s2}{Coordenada normalizada y}\PY{l+s+s2}{\PYZdq{}}\PY{p}{)}

\PY{n}{plt}\PY{o}{.}\PY{n}{figure}\PY{p}{(}\PY{p}{)}
\PY{n}{plt}\PY{o}{.}\PY{n}{imshow}\PY{p}{(}\PY{n}{G\PYZus{}log}\PY{p}{,} \PY{n}{extent}\PY{o}{=}\PY{p}{[}\PY{o}{\PYZhy{}} \PY{n}{N}\PY{p}{,} \PY{n}{N}\PY{p}{,} \PY{o}{\PYZhy{}}\PY{n}{M}\PY{p}{,} \PY{n}{M}\PY{p}{]}\PY{p}{)}
\PY{n}{plt}\PY{o}{.}\PY{n}{title}\PY{p}{(}\PY{l+s+s2}{\PYZdq{}}\PY{l+s+s2}{Holograma filtrado al orden 1 de difracción}\PY{l+s+s2}{\PYZdq{}}\PY{p}{)}
\PY{n}{plt}\PY{o}{.}\PY{n}{xlabel}\PY{p}{(}\PY{l+s+s2}{\PYZdq{}}\PY{l+s+s2}{Coordenada normalizada  x}\PY{l+s+s2}{\PYZdq{}}\PY{p}{)}
\PY{n}{plt}\PY{o}{.}\PY{n}{ylabel}\PY{p}{(}\PY{l+s+s2}{\PYZdq{}}\PY{l+s+s2}{Coordenada normalizada y}\PY{l+s+s2}{\PYZdq{}}\PY{p}{)}
\end{Verbatim}
\end{tcolorbox}

            \begin{tcolorbox}[breakable, size=fbox, boxrule=.5pt, pad at break*=1mm, opacityfill=0]
\prompt{Out}{outcolor}{12}{\boxspacing}
\begin{Verbatim}[commandchars=\\\{\}]
Text(0, 0.5, 'Coordenada normalizada y')
\end{Verbatim}
\end{tcolorbox}
        
    \begin{center}
    \adjustimage{max size={  0.8\linewidth}{  0.8\paperheight}}{output_22_1.png} 
    \end{center}
 
    \vspace{0.7cm}
    \begin{center}
    \adjustimage{max size={  0.8\linewidth}{  0.8\paperheight}}{output_22_2.png}
    \end{center}
     \vspace{0.7cm}
    
    \begin{tcolorbox}[breakable, size=fbox, boxrule=1pt, pad at break*=1mm,colback=cellbackground, colframe=cellborder]
\prompt{In}{incolor}{13}{\boxspacing}
\begin{Verbatim}[commandchars=\\\{\}]
\PY{n}{plt}\PY{o}{.}\PY{n}{figure}\PY{p}{(}\PY{p}{)}
\PY{n}{plt}\PY{o}{.}\PY{n}{subplot}\PY{p}{(}\PY{p}{)}
\PY{n}{plt}\PY{o}{.}\PY{n}{imshow}\PY{p}{(}\PY{n}{CF\PYZus{}log}\PY{p}{,} \PY{n}{cmap} \PY{o}{=} \PY{l+s+s1}{\PYZsq{}}\PY{l+s+s1}{gray}\PY{l+s+s1}{\PYZsq{}}\PY{p}{,} \PY{n}{extent}\PY{o}{=}\PY{p}{[}\PY{o}{\PYZhy{}}\PY{n}{N}\PY{p}{,} \PY{n}{N}\PY{p}{,} \PY{o}{\PYZhy{}}\PY{n}{M}\PY{p}{,} \PY{n}{M}\PY{p}{]}\PY{p}{)}
\PY{n}{plt}\PY{o}{.}\PY{n}{title}\PY{p}{(}\PY{l+s+s2}{\PYZdq{}}\PY{l+s+s2}{Amplitud de la onda reconstruida del objeto}\PY{l+s+s2}{\PYZdq{}}\PY{p}{)}
\PY{n}{plt}\PY{o}{.}\PY{n}{xlabel}\PY{p}{(}\PY{l+s+s2}{\PYZdq{}}\PY{l+s+s2}{Coordenada normalizada x}\PY{l+s+s2}{\PYZdq{}}\PY{p}{)}
\PY{n}{plt}\PY{o}{.}\PY{n}{ylabel}\PY{p}{(}\PY{l+s+s2}{\PYZdq{}}\PY{l+s+s2}{Coordenada normalizada y}\PY{l+s+s2}{\PYZdq{}}\PY{p}{)}
        
\PY{n}{plt}\PY{o}{.}\PY{n}{figure}\PY{p}{(}\PY{p}{)}
\PY{n}{plt}\PY{o}{.}\PY{n}{subplot}\PY{p}{(}\PY{p}{)}
\PY{n}{plt}\PY{o}{.}\PY{n}{imshow}\PY{p}{(}\PY{n}{np}\PY{o}{.}\PY{n}{angle}\PY{p}{(}\PY{n}{CF}\PY{p}{)}\PY{p}{,} \PY{n}{extent}\PY{o}{=}\PY{p}{[}\PY{o}{\PYZhy{}}\PY{n}{N}\PY{p}{,} \PY{n}{N}\PY{p}{,} \PY{o}{\PYZhy{}}\PY{n}{M}\PY{p}{,} \PY{n}{M}\PY{p}{]}\PY{p}{)}
\PY{n}{plt}\PY{o}{.}\PY{n}{title}\PY{p}{(}\PY{l+s+s2}{\PYZdq{}}\PY{l+s+s2}{Fase de la onda reconstruida del objeto}\PY{l+s+s2}{\PYZdq{}}\PY{p}{)}
\PY{n}{plt}\PY{o}{.}\PY{n}{xlabel}\PY{p}{(}\PY{l+s+s2}{\PYZdq{}}\PY{l+s+s2}{Coordenada normalizada x}\PY{l+s+s2}{\PYZdq{}}\PY{p}{)}
\PY{n}{plt}\PY{o}{.}\PY{n}{ylabel}\PY{p}{(}\PY{l+s+s2}{\PYZdq{}}\PY{l+s+s2}{Coordenada normalizada y}\PY{l+s+s2}{\PYZdq{}}\PY{p}{)}
\end{Verbatim}
\end{tcolorbox}

            \begin{tcolorbox}[breakable, size=fbox, boxrule=.5pt, pad at break*=1mm, opacityfill=0]
\prompt{Out}{outcolor}{13}{\boxspacing}
\begin{Verbatim}[commandchars=\\\{\}]
Text(0, 0.5, 'Coordenada normalizada y')
\end{Verbatim}
\end{tcolorbox}
        
    \begin{center}
    \adjustimage{max size={  0.8\linewidth}{  0.8\paperheight}}{output_23_1.png}
    \end{center}
     \vspace{0.7cm}
    
    \begin{center}
    \adjustimage{max size={  0.8\linewidth}{  0.8\paperheight}}{output_23_2.png}
    \end{center}
     \vspace{0.7cm}
    
    \hypertarget{tal-y-como-se-esperaba-el-muxe9todo-de-fourier-pierde-informaciuxf3n-en-la-reconstrucciuxf3n.}{%
\section*{Tal y como se esperaba, el método de Fourier pierde información
en la
reconstrucción.}\label{tal-y-como-se-esperaba-el-muxe9todo-de-fourier-pierde-informaciuxf3n-en-la-reconstrucciuxf3n.}}


    % Add a bibliography block to the postdoc
    
    
    
\end{document}
